\documentclass[10pt,a4paper,ragged2e]{altacv}
\geometry{left=2cm,right=10cm,marginparwidth=6.8cm,marginparsep=1.2cm,top=1.25cm,bottom=1.25cm}
\ifxetexorluatex
  \setmainfont{Carlito}
\else
  \usepackage[utf8]{inputenc}
  \usepackage[T1]{fontenc}
  \usepackage[default]{lato}
\fi
\definecolor{RITOrange}{HTML}{F76902}
\definecolor{Black}{HTML}{000000}
\definecolor{SlateGrey}{HTML}{2E2E2E}
\definecolor{LightGrey}{HTML}{2E2E2E}
\colorlet{name}{RITOrange}
\colorlet{heading}{RITOrange}
\colorlet{accent}{Black}
\colorlet{emphasis}{SlateGrey}
\colorlet{body}{LightGrey}
\renewcommand{\itemmarker}{{\small\textbullet}}
\renewcommand{\ratingmarker}{\faCircle}
\addbibresource{sample.bib}

\begin{document}
\name{JOSEPH VITA}
\tagline{Rochester Institute of Technology}
\personalinfo{
  \email{jcvita2012@gmail.com}
  \phone{631-834-8067}
%  \mailaddress{Address, Street, 00000 County}
%  \location{}
  \github{www.github.com/Jcvita}
%  \homepage{example.com/}
%  \twitter{@joe_life}
  \linkedin{https://www.linkedin.com/in/joseph-vita-082b32191/}
}

%% Make the header extend all the way to the right, if you want.
\begin{fullwidth}
\makecvheader
\end{fullwidth}

%% Depending on your tastes, you may want to make fonts of itemize environments slightly smaller
\AtBeginEnvironment{itemize}{\small}

%% Provide the file name containing the sidebar contents as an optional parameter to \cvsection.
%% You can always just use \marginpar{...} if you do
%% not need to align the top of the contents to any
%% \cvsection title in the "main" bar.
\cvsection[page1sidebar]{Projects}
\cvproject{Python stock data collector and analyzer}
\begin{itemize}
\item Using Python libraries such as TA-Lib to generate graphs and statistics of real time stock data
\smallskip
\item Collaborating with a small group using Git
\smallskip
\item Using machine learning frameworks such as Tensorflow to learn trends and predict buy/sell points in the market and add weights and biases based on updated news trends.
\smallskip
\item Using Docker to create optimized environments for MongoDB and Flask services
\end{itemize}
\smallskip
\smallskip
\cvproject{snip2clip}
\begin{itemize}
\item Used Visual Studio 2019 with C\# to make my own version of the Windows snipping tool
\smallskip
\item Used the Tesseract API to run OCR on the snipped images
\smallskip
\item Created a tool that myself and others use on a daily basis and works more efficiently than Microsoft's current snipping tool
\end{itemize}
\smallskip
\smallskip
\cvproject{AutoDJ}
\begin{itemize}
\item Kahoot-style voting software for music selection tailored towards large groups
\smallskip
\item Collaborated with a small group using Git to build a full stack application
\smallskip
\item Worked with backend libraries such as Node.js and Express
\smallskip
\item Managing server hardware to set up working environments for MongoDB
\end{itemize}
\smallskip
\smallskip
\cvproject{Handwritten Digit Classifier}
\begin{itemize}
\item Used TensorFlow and Keras to classify handwritten digits in 28x28 images with a ~95\% accuracy
\smallskip
\item Taught myself the basics of machine learning and convolutional neural networks
\smallskip
\item Learned how to understand large dataset collection and classification 
\end{itemize}
\cvproject{}

\cvsection{TECHNICAL SKILLS}
\smallskip
\begin{itemize}
\item Python, Java, C\#, C, Javascript, Jquery \& AJAX, Flask, HTML5, CSS, Node.js
\smallskip
\item Git and Github, Microsoft Windows, Office,  Visual Studio, Ubuntu, Kali, Bash, Docker, Docker-Compose, MongoDB
\smallskip
\end{itemize}

%\cvsection{PERSONAL SKILLS}
%\smallskip
%\begin{itemize}
%\item Having leadership qualities.
%\smallskip
%\item Ability to work under pressure.
%\smallskip
%\item Comfortable working independently.
%\smallskip
%\item Ability to collaborate as a team and resolve conflicts 
%\end{itemize}

\clearpage

% \cvsection[page2sidebar]{Publications}

\nocite{*}

% %% If the NEXT page doesn't start with a \cvsection but you'd
% %% still like to add a sidebar, then use this command on THIS
% %% page to add it. The optional argument lets you pull up the
% %% sidebar a bit so that it looks aligned with the top of the
% %% main column.
% % \addnextpagesidebar[-1ex]{page3sidebar}


\end{document}
